\section{Overview}
\label{sec:overview}

\subsection{Threat Model}
\label{sec:threatmodel}

Kernel TOCTOU is a local privilege escalation vulnerability. The vulnerability could allow local users or malicious software to gain full root privileges. We assume an attacker has a user account that can upload and run arbitrary programs with user privilege, or he can access such a program. The attacker has arbitrary memory read and writes primitives. He is also able to call any system services or load any library. The DEP policy (\textbf{W}rite $\oplus$ e\textbf{X}ecute) and ASLR is not necessary. We assume the attacker has full knowledge about the system kernel, including the memory layout. However, he can not read or write any kernel memory as a classical operating system would not allow. The attacker aims at running arbitrary code in the kernel, hence obtain the highest privilege. 

We aim at the Windows OS. An operating system is very complicated. Without investigation,  we can not be sure that the mitigation will work on other OS, even the underlying mechanism should work. Our mitigation will not work on the Linux kernel because it already utilizes the Intel CPU feature SMAP in a conventional way. We will elaborate on this in~\autoref{sec:background}.

Considering we leverage a hardware feature from Intel CPU, The host system should use Intel CPU with SMAP capability.
