%! TEX root = '../main.tex'

\section{Introduction}
\label{sec:ktoctou-introduction}

Time-of-check-to-time-of-use (TOCTOU), a.k.a race condition, or double fetch, is a long-standing issue in software security. As the name implied, it involves two references on the same variable or system state at different timing. The attacker usually passes the first security check with a benign value and then replace it with a malicious one before the second reference. This behavior could introduce faults into the system, therefore empower the attack ability to exploit the system.

TOCTOU vulnerabilities are inevitable at the kernel level. Specifically, modern operating systems such as Windows and Linux separate the kernel and user programs into two security domains. The kernel runs in the privileged domain, and the user programs run in the unprivileged domain. Due to the domain separation, the kernel serves the user programs on a service-client basis through system calls, and it unavoidably gets parameters from the userspace. Reading the same user-mode variable repeatedly by the kernel may lead to data inconsistency under a race condition between the kernel and user programs. In specific, the user program invokes a system call with the valid parameters and then replaces them with the malicious values after the security check passing. In this way, a user program can inject faults into the kernel, which could lead to the kernel malfunction or even running user-provided unsafe code. In particular, attackers could utilize such vulnerabilities to bypass sandbox protection of browsers or further get the administrator privilege after having a local user account.

The kernel-level TOCTOU vulnerabilities widely exist in Microsoft Windows. Notably, a graphical subsystem kernel module tightly coupled with user-mode libraries freely access user data structures, among which double fetches on the same variable is not unusual. In essence, the kernel repeatedly reads the same user memory address within the same system call.

To find the bugs with the typical memory-access pattern, we develop \toolname, a fuzzing tool that leverages an Intel processor feature, Supervisor Mode Access Prevention (SMAP), and then combine it with a run-time hypervisor to monitor the Windows kernel efficiently. SMAP is introduced since the Intel Broadwell microarchitecture. It prevents the kernel from freely accessing userspace so that such access will raise an exception. It accurately serves the purpose of notifying us when the kernel access a user-mode address. For each system call, we pick out those user-mode addresses that the kernel reads twice, which are the candidates for further analysis. 

Furthermore, we present \name, to the best of our knowledge, it is the first run-time protection for kernel-level TOCTOU vulnerabilities. It also leverages the same hardware feature SMAP and the hypervisor. Base on our observation of the vulnerabilities, to retain the virtual memory that the kernel fetched unwritable is the key to successful protection. Considering hardware capability and practical aspects, we propose the following idea. Whenever the kernel accesses a user-mode memory, \name protects the corresponding page by setting it as a kernel page until the current system call ends so that no other user threads can tamper with it. We also solve the practical issue that benign read and write to the data that resides on the protected page.


Due to the Windows system's complex nature and the fact that it did not adopt SMAP, the amount of exceptions is enormous when enabling this feature, and the causes are various. It is difficult to handle such a multifactorial situation, if not impossible. To first solve the core issue without interference from other factors, we use the hypervisor to confine SMAP within specific processes. The hypervisor takes action on the process context switch event and makes SMAP active only when the specific process is running on the processor. Later, we find that to prevent nested SMAP exceptions from forming a deadlock during page table walking, isolating SMAP is also crucial. Additionally, it makes \name more configurable, avoids unnecessary processes, thus improves performance. Therefore, we consider the light-weight hypervisor framework as one of the contributions of this paper.

\name successfully mitigates real-world vulnerabilities such as CVE-2008-2252 and the family of CVE-2013-1254. It prevents the race condition from happening so that the system can operate normally during the attacks. We evaluate \name and the light-weight hypervisor with 18 benchmark programs and real-world applications. Our evaluation results show that \name imposes little extra overhead (less than 10\% on average).

\textbf{Contributions.} To summarize, we make the following contributions in this paper:
\begin{itemize}[leftmargin=*]
    \item We identify kernel TOCTOU vulnerability using study cases, and demonstrated with practical exploits against them. 
    \item We present \name, a novel run-time mitigation framework leveraging hardware feature (SMAP). 
    \item We propose a configurable light-weight hypervisor to isolate system-wide processor features.
    \item We develop a fuzzing tool \toolname for detecting kernel TOCTOU vulnerabilities.
    \item We have implemented \name and evaluated it with a number of benchmark programs with real-word vulnerabilities.
  
\end{itemize}

\textbf{Responsible disclosure}
We have reported our finds to Microsoft. 

\textbf{Roadmap}
The rest of this paper is organized as the following. 
In~\autoref{sec:ktoctou-background}, we provide necessary background related to the mechanism behind kernel TOCTOU vulnerabilities and SMAP.~\autoref{sec:ktoctou-overview} describes the objectives, threat model and scope, challenges and architecture of \name.~\autoref{sec:ktoctou-experiment} shows the vulnerability findings on Windows with our fuzzing tool, we present how we perform run-time mitigation with SMAP and a light-weight hypervisor (~\autoref{sec:ktoctou-design}), with implementation details (~\autoref{sec:ktoctou-implementation}), and evaluation (~\autoref{sec:ktoctou-evaluation}) respectively.  In~\autoref{sec:ktoctou-discussion}, we discuss an alternative to solve the writing conflicts and methods of fuzzing, followed by related work in~\autoref{sec:ktoctou-relatedwork}, we discuss related works. Finally~\autoref{sec:ktoctou-conclusion} concludes the paper.
