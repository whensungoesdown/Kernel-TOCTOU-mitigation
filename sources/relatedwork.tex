\section{Related Work}
\label{sec:ktoctou-relatedwork}




%\textbf{Transactional Memory}~\cite{herlihy1993transactional}~\cite{shavit1997software}~\cite{rajwar2012intel}~\cite{haring2012ibm}~\cite{jacobi2012transactional}~\cite{click2009azul}~\cite{dice2009early} attempts to simplify concurrent programming by allowing memory read and write instructions to be executed in an atomic way. Usually memory access transactions are short-lived activities only when accessing a small memory area, such as a shared counter or shared double-linked list. Since transactional memory is limited resource, it's not practical to cover large portion of code. 
%
%In order to cover large region of code, one way is to leverage the lightweight transaction mode of hardware-assistant transactional memory~\cite{gupta2009using}, the other is to use software transactional memory\cite{kestor2014trex}. Or combine both hardware and software method together~\cite{zhang2016txrace}. Those projects are mostly focusing bug hunting. To deploy those as mitigations, transaction primitives are needed to add to the source code either by programmers or compilers.
%
%The idea behind bug hunting projects is similar to ours, which is marking a conflict zone and then capturing conflicts. In our case, almost the whole life-span of a system call is marked as the conflict zone. Then hardware feather SMAP and page fault handling mechanism are leveraged to capture and handle conflicts.


\textbf{\textit{Transactional Memory.}} As discussed in~\autoref{sec:ktoctou-background}, the root cause of kernel-level TOCTOU is that both kernel and user access the same memory address, which produce data inconsistency. Transactional memory~\cite{shavit1997software}~\cite{rajwar2012intel}~\cite{herlihy1993transactional} is a mechanism that allows a series of memory operations to execute atomically. It is an intuitive method for solving data inconsistency problems. Hardware transactional memory has become available on Intel processors since the Haswell macroarchitecture~\cite{rajwar2012intel}. However, even before the hardware transactional memory feature is widely available, researchers use software transactional memory~\cite{kestor2014trex}~\cite{abadi2008semantics} to detect race conditions. It is more of a static analysis based approach.


With the hardware support, TxRace~\cite{zhang2016txrace} detects data race at run-time. It instruments a multithreaded program to transform synchronization-free regions into transactions and leverage the conflict detection mechanism of Intel Restricted Transactional Memory (RTM). However, the hardware's limitation is also apparent.  Intel RTM does not support arbitrarily long transactions, simply aborting any transactions exceeding the hardware buffer's capacity for transactional states. It can only run a short range of code and does not support a wide variety of processor mechanisms such as system calls, interruptions, special instructions (CPUID, MMX), IPI. Any of those and even access invalid memory may cause a transactional abort. Therefore, the hardware transactional memory is not suitable as run-time mitigation for kernel-level TOCOTU but a race condition detector. Although the idea of bug hunting is quite comparable to ours, similarly, we mark a whole system call as a transaction. Once we protect a user page during the system call, other user accesses cause the conflicts. 




\textbf{\textit{Dynamic Binary Instrumentation.}}  It is a common method for monitoring a program's behavior. Tools such as Intel Pin~\cite{luk2005pin}, DynamoRIO~\cite{nethercote2007valgrind} injects instrucmentation stubs into the program. It is a technique that widely used in data race detectors~\cite{savage1997eraser}~\cite{o2003hybrid}~\cite{yu2005racetrack}~\cite{bond2010pacer}~\cite{marino2009literace}~\cite{flanagan2009fasttrack}~\cite{pozniansky2007multirace}. However dynamic binary instructation is only available for user-mode programs and the high performance overhead is the biggest limitation. 


\textbf{\textit{Hardware data breakpoints.}} DataCollider~\cite{erickson2010effective} leverage software code breakpoints and hardware data breakpoints to efficiently detecting data races in kernel modules.  It randomly samples a small percentage of memory accesses as candidates for data-race detection. First, it uses static analysis to decide the sampling set, which are the instruction locations that access memory. Then it inserts code breakpoints, and when one code breakpoint fires, it uses a data breakpoint to trap the second access to detect conflicts. However, data breakpoint is a limited resource on commercial processors. An Intel processor usually has four hardware data breakpoints. Thus, it can only monitor a few program locations simultaneously. It is sufficient for hunting data race bugs, but not enough hardware resources for run-time kernel-level TOCTOU mitigation.

