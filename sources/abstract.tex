%! TEX root = 'main.tex'
\section{Abstract}

Time-of-check-to-time-of-use (TOCTOU) attacks become popular in the hacker/security researcher community. It is a long-history issue with many forms. In this paper, we address this issue that happens in the operating system kernel. If the kernel reference a user-mode variable more than one time during a system call. It is possible that a malicious user changes the variable deliberately. So that the kernel lost data inconsistency. The value differences could lead to a serious vulnerability such as buffer overflow.  We examined the root cause of real-world cases. And we present the first run-time mitigation for kernel TOCTOU vulnerability. We take advantage of an Intel processor feature Supervisor Mode Access Prevention (SMAP). The CPU will raise an exception when the kernel access user space. It enables us the ability to identify the kernel's dangerous behavior with low-overhead. We take over the page fault handler to handle the exception. When it comes, we set the page's attribute so that it becomes a kernel page for the rest of the system call.  Thus, no user code can access that variable. The related page sharing issues also solved with a novel solution. We also develop a lightweight hypervisor. Its purpose is to contain a system-wide hardware feather within a particular process.  This makes our mitigation more flexible and practical. Because of the hardware feature and the hypervisor, the performance overhead is modest. We test it with non-trivial applications. On average, it's below 5\%
