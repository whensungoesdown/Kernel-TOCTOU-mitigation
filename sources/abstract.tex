%! TEX root = 'main.tex'
\section{Abstract}

%Time-of-check-to-time-of-use (TOCTOU) attacks have been a key threat to systems security. It is a long-history issue with many forms. In this paper, we address this issue that happens in the operating system kernel. If the kernel reference a user-mode variable more than one time during a system call. It is possible that a malicious user changes the variable deliberately. So that the kernel lost data consistency. The value differences could lead to a serious vulnerability such as buffer overflow.  We examined the root cause of real-world cases. And we present the first run-time mitigation for kernel TOCTOU vulnerability. We take advantage of an Intel processor feature Supervisor Mode Access Prevention (SMAP). The CPU will raise an exception when the kernel access user space. It enables us the ability to identify the kernel's dangerous behavior with low-overhead. We take over the page fault handler to handle the exception. When it comes, we set the page's attribute so that it becomes a kernel page for the rest of the system call.  Thus, no user code can access that variable. The related page sharing issues also solved with a novel solution. We also develop a lightweight hypervisor. Its purpose is to contain a system-wide hardware feather within a particular process.  This makes our mitigation more flexible and practical. Because of the hardware feature and the hypervisor, the performance overhead is modest. We test it with non-trivial applications. On average, it's below 10\%

Kernel-level time-of-check-to-time-of-use (TOCTOU) widely exists in operating systems, especially Microsoft Windows. When serving a system call, the kernel inevitably gets parameters from the userspace. Read the same user-mode variable repeatedly may lead to data inconsistency under a race condition between the kernel and userspace. This paper presents \name, an efficient run-time mitigation technique on Windows, to prevent the double fetch behavior without recompiling it from source code or modifying the kernel's binary. The core of \name is to use \texttt{Supervisor Mode Access Prevention (SMAP)}, a hardware feature, to detect kernel access to userspace. To leverage SMAP, \name customized the kernel xxxxx. We investigated and found one, if not the only solution, to make the system recover from the fatal SMAP exception. \hb{Dont understand what is the purpose of this sentence }
We further develop a new technique, a lightweight hypervisor where one process contains a system-wide CPU feature, to improve the flexibility and performance of \name. We evaluate \name and lightweight hypervisor with 18 benchmark programs and real-world applications. Our evaluation results show that \name impose little extra overhead(less than 10 \% on average).
