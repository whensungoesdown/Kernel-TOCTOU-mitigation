%! TEX root = 'main.tex'

\section{Evaluation}
\label{sec:ktoctou-evaluation}

This section, we evaluate \name's performance overhead and how it protects the operating system from real-world vulnerabilities.


\subsection{Performance}
In this section, we focus on performance evaluation.
As previously mentioned, \name has two key components, the system module, and the hypervisor. To present the performance impact introduced by this mitigation, we conduct the tests in three parts. 

All tests run on the PC with Intel Core i5-6400 (6th Gen CPU Skylake), ASUS H110M-C motherboard (Intel H110 Chipset, Realtek RTL8111H Network Controller), 8GB RAM, and 500GB hard disk.

\textbf{\textit{Benchmarks.}} The hypervisor plays an essential part in \name. Without the hypervisor confining the SMAP feature, it would not be possible to debug a complex system like Windows with a system-wide unsupported feature. Nowadays, a hypervisor is part of the cloud computing infrastructure and can be regarded as a built-in component. The Windows 10 operating system even brings its native hypervisor to deliver security goals. Under those circumstances, \name can be added to the existing hypervisor with less performance overhead imposed on the system. 


To evaluate the hypervisor's performance overhead, we use the well-known benchmark SPEC 2006. We understand that this benchmark is for processor instruction set evaluation, specifically for microarchitectural aspects, such as instruction execution, branch prediction accuracy, cache policies. We choose several programs from the set. They are all non-GUI and computational intensive programs. Therefore the performance overhead incurred is primarily due to the hypervisor. Although our hypervisor and Windows HVCI have different objectives, we compare them to show that run-time protection utilizes virtualization techniques is practical.

\begin{figure}[th]
  \includegraphics[width=0.47\textwidth]{figures/benchmark3}
  \centering
  \caption{Performance overhead on the SPEC benchmarks incurred by the run-time load hypervisor. HVCI represent the Windows 10 native hypervisor for Hypervisor-Protected Code Integrity. All overheads are normalized to the unprotected system running benchmark.}
  \label{fig:ktoctou-benchmark}
\end{figure}


\autoref{fig:ktoctou-benchmark} shows that hypervisor's performance overhead is acceptable, on average, 3.25\%. HVCI yields a modest performance overhead of 0.81\%.

% show my respect.. maybe remove it later

Our hypervisor is slower than HVCI, particularly in two benchmark programs. Learning more about computer architecture and virtualization techniques, we wish to improve our hypervisor to perform better.


\textbf{\textit{Non-trivial Applications.}} We also evaluate \name on several non-trivial applications, as shown in~\autoref{fig:ktoctou-performance}. The applications we choose are meaningful because the kernel-TOCTOU vulnerability may threaten them in real-world scenarios. For example, a web server such as Nginx normally runs in a non-root account and takes external requests.

To test web servers, we count their response time for a web page request. For compression software, the test is to compress large files. We use speedtest1.c, which is a performance testing program for sqlite3. The result shows that the performance overhead is acceptable.

\begin{figure}[th]
  \includegraphics[width=0.47\textwidth]{figures/performance4}
  \centering
  \caption{Performance overhead in non-trivial applications. Overhead mostly being introduced on system calls that need to fetch user parameters}
  \label{fig:ktoctou-performance}
\end{figure}





\textbf{\textit{GUI.}} Through our investigation, we find that the Windows graphical subsystem, namely, Win32k.sys, has the most double-fetch issues.  Since GUI programs need to redraw their graphical components, they invoke the win32k system calls in a high frequency, even for a simple program such as Windows notepad. Therefore,  our mitigation incurs an unneglectable performance over GUI programs. The overhead is primarily affected by how often the interface refreshes. For example, if minimizing a GUI program's window, its performance will not be slow down by the graphical interface at all. 

We compare GUI programs with non-GUI programs in the following aspects: the number of user pages accessed by kernel per system call and how many double fetches occur. We drag the GUI program's window to trigger redraw, and we send one URL request to the web servers per second. The measurement takes the first 500 system calls.

\vspace*{-\baselineskip}
\begin{center}
\begin{table}[ht]
	\small
	%\renewcommand{\arraystretch}{1.3}
	\caption{System call count and user-pages accessed for GUI \& non-GUI programs }
	\label{table:pages}
	\centering
%	\begin{tabular}{ l|p{0.04\textwidth}|p{0.108\textwidth}|l|p{0.045\textwidth}|p{0.045\textwidth} } 
%	\begin{tabular}{ c@{}|@{}c@{}|@{}c@{}|@{}c@{}|@{}c@{}|@{}c }   
	\begin{tabular}{@{}>{\centering\arraybackslash}m{1.40cm}@{}|
			@{}>{\centering\arraybackslash}m{1.15cm}@{}|
			@{}>{\centering\arraybackslash}m{2.30cm}@{}|
			c|
			@{}>{\centering\arraybackslash}m{1.15cm}@{}|
			@{}>{\centering\arraybackslash}m{0.97cm}@{} } 
		\hline
		Programs & System Calls & Protected Pages(r, w) & \textbf{avg.} & Double Fetch & Time (ms)\\ 
		\hline
		nginx & 500 & 711(711, \textbf{0}) & 1.42 & \textbf{223} &12312\\ 
		apache & 500 & 689(689, \textbf{0})  & 1.38 & \textbf{205} &11339\\ 
		notepad & 500 & 1434(1102, 241) & 2.87 & \textbf{1373} & 1859 \\ 
		freecell & 500 & 1352(1165, 187) & 2.70 & \textbf{1266} & 1500 \\ 
		\hline
	\end{tabular}
\end{table}
\end{center}
\vspace*{-\baselineskip}


\autoref{table:pages} shows some interesting results. Refreshing the GUI takes tremendous system calls. As expected, both the kernel and the Win32k subsystem accesses user pages, but the win32k accesses more than the kernel does. We find that they both \textit{capture} the user parameters at the beginning of each system call through reverse engineering, but the win32k module read/write user data even in the middle of a system call.


We also count the number of reads and writes on user pages. For non-GUI programs, the number of writes is zero, which is strange because most system calls need to write results back to the user program. With investigation, the causes are as follows. Windows provides three methods to transfer data between system calls and user programs, namely, Buffered I/O, Direct I/O, and Neither Buffered Nor Direct I/O. Among the three, the kernel mostly uses Buffer I/O and Direct I/O, which do not need to write to user-mode buffer directly. However, the kernel still needs to write user-mode variables such as the filehandle in system call \texttt{NtCreateFile()}.~\autoref{fig:probecode} shows the pseudo-code similar to what the kernel uses to validate user parameters. We can see that the code always reads the variable first and the SMAP exception only captures first access. Therefore, this coding style is another cause for the zero writes. The result shows that the kernel is well regulated on accessing user data. On the other side, the Win32k module has many writes, which tell a different story.

\begin{figure}[th]
  \includegraphics[width=0.47\textwidth]{figures/probecode}
  \centering
  \caption{Pseudo-code for validating a file handler.}
  \label{fig:probecode}
\end{figure}


% uty: vsapce not suitable for figures, only for tabular
%\vspace*{-10mm}

% for later reference
%Buffered I/O
%The operating system creates a nonpaged system buffer, equal in size to the application's buffer. The I/O manager copies user data into the system buffer for write operations before calling the driver stack. The I/O manager copies data from the system buffer into the application's buffer after the driver stack completes the requested operation for read operations.
%
%Direct I/O
%The operating system locks the application's buffer in memory. It then creates a memory descriptor list (MDL) that identifies the locked memory pages and passes the MDL to the driver stack. Drivers access the locked pages through the MDL.
%
%Neither Buffered Nor Direct I/O
%The operating system passes the application buffer's virtual starting address and size to the driver stack. The buffer is only accessible from drivers that execute in the application's thread context.
%



The column \textit{Double Fetch} lists the user addresses that are read more than once during individual system calls. We trace this information with the \toolname as previously introduced in~\autoref{sec:ktoctou-experiment}. Many of those double-fetch records are duplicates and benign. For example, the kernel needs to read data from Process Environment Block (PEB) or Thread Environment Block (TEB), or data structures located in userspace.  However, the Win32k module has many more cases. ~\autoref{table:doubleread} gives a glimpse of them. Every two rows indicate a double-fetch case, where after the first read, the subsequent instruction revisits the same address shortly after, and the identical CR3 and TEB show that two fetches are from the same thread of the same process.

Furthermore, we find that the Win32k module directly read the user variable not within a try-catch block in some of those cases. It is dangerous. The user programs can free the user memory and cause the kernel code to access an invalid page without protection, which leads to a kernel crash.


\vspace*{-\baselineskip}
\begin{center}
	\begin{table}[ht]
		\small
		%\renewcommand{\arraystretch}{1.2}
		\caption{In the selected double-fetch results, for every two lines, they have the same CR3 and TEB, which indicate the two records are from the same process, same thread. The two EIP are shortly apart, but the addresses they reference are the same user-mode address.}
		\label{table:doubleread}
		\centering
		%\begin{tabular}{ l l l l }
		\begin{tabular}{@{}>{\centering\arraybackslash}m{2.05cm}@{}|
				@{}>{\centering\arraybackslash}m{2.05cm}@{}|
				@{}>{\centering\arraybackslash}m{2.05cm}@{}|
				@{}>{\centering\arraybackslash}m{2.05cm}@{} } 
			\hline
			Cr3 & Eip & Addr. & Teb \\
			\hline
			0x6d40320 & 0xbf812de4 & \textbf{0x4808c4} & 0x7ffdd000 \\
			0x6d40320 & 0xbf812e4b & \textbf{0x4808c4} & 0x7ffdd000 \\
			\hline
			0x6d40320 & 0xbf812dea & \textbf{0x4808c8} & 0x7ffdd000 \\
			0x6d40320 & 0xbf812e55 & \textbf{0x4808c8} & 0x7ffdd000 \\
			\hline
			0x6d40320 & 0xbf812daf & \textbf{0x480750} & 0x7ffdd000 \\
			0x6d40320 & 0xbf812e21 & \textbf{0x480750} & 0x7ffdd000 \\
			\hline
			0x6d40320 & 0xbf80c04d & 0x7ffdd206 & 0x7ffdd000 \\
			0x6d40320 & 0xbf812ebe & 0x7ffdd206 & 0x7ffdd000 \\
			\hline
		\end{tabular}
	\end{table}
\end{center}
\vspace*{-\baselineskip}



%Part of the overhead is introduced due to the overall intercepting of page fault exceptions of the system. The page fault handler is called in high frequency. Our page fault hook currently is installed directly in the IDT table of each processor. Hence every page fault exception goes through our handler. Even though, in the very begining, we pass through exceptions that doesn't belong to the target process, still extra instructions are executed for each exception.

%Although we use virtualization techniques, but the hypervisor we use is a very simple one. Unlike commercial virtualization solutions such as VMWare, Xen and Qemu+Kvm, ours doesn't emulate any hardware devices nor intercept further page mapping translate that between host and guest. Only several types of VM-Exit is inevitable such as control register accessing which we do need to handle it too.



To promote the performance of \name, it would be helpful if  the unnecessary SMAP exceptions during parameter validating can be eliminated. As was done in the Linux kernel, the SMAP feature is temporarily disabled during \texttt{copy\_from\_user()} and \texttt{copy\_to\_user()}, the gateway functions. In Windows kernel, \texttt{ProbeForRead()} and \texttt{ProbeForWrite()} are the primary cause of SMAP exception. However, such \textit{probe} has many variants such as \texttt{ProbeAndWriteHandle()}, \texttt{ProbeForWriteIoStatus()} and they are only partial of the user-data-copying code. Some of them are macros instead of functions, making it difficult to fix them without recompiling the kernel. 

\subsection{Case Study}


\textbf{\textit{CVE-2008-2252}} is a kernel-level TOCTOU vulnerability reported by Thomas Garnier in 2008 and patched in Microsoft security bulletins ms08-061.  It has been analyzed by many research works~\cite{wang2019dftracker}~\cite{jurczyk2013identifying}. To evaluate the effectiveness of \name, we test it on a real hardware machine. Since SMAP and SMEP are only available on a relatively new processor, we have to install an old system on a modern PC. 

%It is troublesome to install Windows XP because the chipset with an integrated hard disk controller no longer supports IDE mode, and Windows XP does not support the advanced host controller interface (AHCI) either. For us, we need an additional AHCI driver for Windows XP and a tool to virtualize a floppy drive~\cite{installxpskylake} to provide it during the installation.


We write a program to exploit the CVE-2008-2252 vulnerability. To create a buffer overflow in the kernel, we need to enlarge the user-mode variable between the two kernel reads.  The exploit creates two threads. Thread zero first allocates a virtual page at address zero for the parameters, which is necessary to bypass the sanity check of the system call \texttt{NtUserMessageCall()}. Then, it repeatedly calls the upper layer Win32 API \texttt{SendMessage()} (\texttt{WM\_COPYDATA}) with the malicious parameters to open the attack time window. \texttt{SendMessage()} calls a lower layer function \texttt{NtUserMessageCall()}, which eventually calls the vulnerable win32k internal function \texttt{xxxInterSendMsgEx()}. Simultaneously, thread one keeps flipping the high bit of the target variable on page zero.

\begin{figure}[th]
  \includegraphics[width=0.47\textwidth]{figures/ms08061case2}
  \centering
  \caption{The attack thread tries to flip the data within the attack window, between instruction \texttt{\textcircled{2}} and \texttt{\textcircled{3}}. The kernel first touches the page at instruction \texttt{\textcircled{1}}, then the mitigation protects it afterward until the end of the current system call, creating a larger page-protect-window. As soon as the attack thread writes the protected page, the page fault handler suspends it until the protection ends.}
  \label{fig:ms08061case}
\end{figure}



As shown in~\autoref{fig:ms08061case}, the kernel reads the user-mode variable (address 0x4) at \texttt{\textcircled{2}} and \texttt{\textcircled{3}}. Between the two instructions is the attack window within which thread one tries to enlarge the variable. To successfully mitigate the attack, we need to ensure this page remains unchanged during the time. The protection starts at \texttt{\textcircled{1}}, several instructions before \texttt{\textcircled{2}}, where the CMP instruction first reads the page hence raise an SMAP exception. The protection continuously effective until the current system call \texttt{NtUserMessageCall()} ends. It covers the entire attack window. When thread one tries to tamper with the variable, it inevitably triggers a page fault regarding privilege violation. The page fault handler then suspends thread one until the system call ends, making it miss the opportunity.



\textbf{\textit{CVE-2013-1254.}} Similar to the vulnerability above mentioned, CVE-2013-1254~\cite{CVE-2013-1254} is another typical TOCTOU vulnerability. It is a family of 27 distinct vulnerabilities~\cite{ms13016}~\cite{jurczyk2013identifying} in the win32k module. It affects a variety of operating systems from Windows XP to Windows Server 2012.

\begin{figure}[th]
  \includegraphics[width=0.47\textwidth]{figures/cve-2013-1254}
  \centering
  \caption{The family of the 27 vulnerabilities in the win32k module is mostly around the parameter security check. The checking code with \texttt{\_W32UserProbeAddress} makes sure that the parameter is from userspace. It is a classic time-of-check-to-time-of-use. When checking, the value is benign where \texttt{[exc+8]} is less than \texttt{\_W32UserProbeAddress}. After that, the attack can replace it with a malicious one before the second read.}
  \label{fig:cve-2013-1254}
\end{figure}



These vulnerabilities share a similar pattern. The code is part of the parameters security check, around \texttt{\_W32UserProbeAddress}, which is the highest possible address for user-mode data. ~\autoref{fig:cve-2013-1254} shows that the flawed code first compares it with the passed in parameters' address, which is \texttt{ecx+8}. Only if the address is legit will the kernel uses it. However, after passing the security check, the kernel mistakenly reread the address. The attacker can abuse this, first pass the security check, then replace the address with a malicious one, such as a kernel-mode address, which may lead to an arbitrary kernel write vulnerability depending on the case.

The protection takes effect when the instruction \texttt{cmp [ecx+8], eax} triggers the SMAP exception. The page is protected, and the kernel gets the same value at the second read.
